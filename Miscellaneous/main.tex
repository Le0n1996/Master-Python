\documentclass[11pt]{article}
\usepackage[utf8]{inputenc}
\usepackage[russian]{babel}
\usepackage{amsmath}
\usepackage{amssymb}
\usepackage{dsfont}
\usepackage[dvipsnames]{xcolor}
\usepackage{setspace}
\usepackage{multirow}
\usepackage[a4paper, outer=1.5cm, inner=1.5cm, top=1cm, bottom=1cm]{geometry}
\usepackage[draft]{graphicx}
\graphicspath{{images/}}

%\parindent=0pt
\newcounter{class}
\newcounter{num}
\newcounter{number}

\newcommand{\n}[1]{\vspace{4mm}\par
\textbf{\textnumero\arabic{num}.}
\stepcounter{num}}

\newcommand{\nast}[1]{\vspace{4mm}\par
\textbf{\textnumero\arabic{num}$\vphantom{1}^\ast$.}
\stepcounter{num}}

\newcommand{\nasst}[1]{\vspace{4mm}\par
\textbf{\textnumero\arabic{num}$\vphantom{1}^{\ast\ast}$.}
\stepcounter{num}}

\setcounter{class}{6}
\setcounter{num}{1}
\setcounter{number}{0}

\pagestyle{empty}

\begin{document}

{\centering\section*{77 задач на лето}}
\vspace*{1cm}

{\centering\subsection*{\textcolor{Emerald}{Часть I:}}}
{\vspace{-2mm}\centering\subsection*{\textit{Числа, делимость, и цифровая запись чисел. (Номера 1-15)}}}

\setcounter{number}{72} \n

%$\!$[\arabic{class}.\arabic{number}] 
Поставь в записи числа 4$\ast$651 вместо $\ast$ такую цифру, чтобы получилось число, которое при делении на 3 даёт в остатке 1.

\setcounter{number}{77} \n

В числе $7030506$ замени все нули одной и той же цифрой так, чтобы полученное число делилось на 9.

\setcounter{number}{81} \n

Найди наибольшее пятизначное число, которое делится на 2 и на 3.

\setcounter{number}{212} \n

Число яблок в корзине --- двузначное. Яблоки можно поровну разделить между 2, 3, или 5 детьми, но нельзя разделить поровну между 4 детьми. (резать яблоки нельзя)\\
Сколько яблок в корзине? 

\setcounter{number}{213} \n

Число груш в ящике меньше 200. Их можно разделить поровну между 2, 3, 4, 5, или 6 детьми. Сколько груш в ящике?

\setcounter{number}{218} \n

Три велосипедиста с общего старта начали движение по круговой дорожке. Первый делает полный круг за 21 минуту, второй --- за 35 мин, третий --- за 15 мин. Через сколько минут они еще раз окажутся вместе на линии старта?

\setcounter{number}{172} \n

По всем вагонам поезда поровну разместили 737 туристов. Сколько было вагонов и сколько туристов в каждом вагоне?

\setcounter{number}{86} \nast

а) Делится ли $39\cdot737+39\cdot281-39\cdot296$ на 19?\smallskip\\
\hspace*{1.55cm} б) Делится ли $38\cdot756+38\cdot239-38\cdot281$ на 17?

\setcounter{number}{100} \nast

К числу 10 припиши слева и справа по одной цифре так, чтобы полученное число делилось на 72.

\setcounter{number}{181} \nast

Докажи, что если в трёхзначном числе средняя цифра равна сумме крайних, то число кратно 11.

\setcounter{number}{176} \nast

К двузначному числу приписали такое же число. Может ли образовавшееся четырёхзначное число быть простым?

\setcounter{number}{111} \nast

Найди число $N>1000$, которое делится на 4, а при делении на 3 дает остаток 2.

\setcounter{number}{853} \nast

Цифру 9, с которой начиналось трёхзначное число, перенесли в конец числа. \\
В результате получилось число на 216 меньше. Какое число было в начале?

\setcounter{number}{48} \nast

Докажи или опровергни утверждение:\\
Разность между трёхзначным числом и суммой его цифр всегда делится на 9.

\setcounter{number}{130} \nasst

Известно, что 3, 5, 7 --- три последовательных простых нечётных числа. Есть ли еще среди натуральных чисел три последовательных нечётных простых числа? Почему?

\pagebreak

{\centering\subsection*{\textcolor{orange}{Часть II:}}}
{\vspace{-2mm}\centering\subsection*{\textit{Дроби, части, уравнения с ними, и не только. (Номера 16-35)}}}

\setcounter{number}{288} \n

В пакете лежали яблоки. Сначала из него взяли половину всех яблок без пяти, а затем $\frac13$ оставшихся яблок. После этого в пакете осталось 10 яблок. Сколько яблок было в пакете?

\setcounter{number}{342} \n

Два тракториста вспахали поле за 6 ч совместной работы. Первый тракторист мог бы один выполнить ту же работу за 10 ч. За сколько часов второй тракторист может вспахать поле?

\setcounter{number}{401} \n

Когда турист прошёл $\frac{3}{10}$ всего пути, то до середины пути ему осталось пройти ещё $4\frac12$ км. Найти длину всего пути.

\setcounter{number}{279} \n

Расстояние между двумя городами турист проехал за два дня. В первый день он проехал половину пути и еще 24 км. Во второй день ему осталось проехать расстояние в три раза меньшее, чем в первый день. Найти расстояние между городами.

\setcounter{number}{290} \n

На столе лежало несколько тетрадей. Когда со стола взяли половину всех тетрадей и ещё одну тетрадь, то осталось 3 тетради. Сколько тетрадей лежало на столе первоначально?

\setcounter{number}{409} \n

Из двух пунктов выехали одновременно навстречу друг другу два велосипедиста. Первый велосипедист может проехать расстояние между пунктами за 6 ч, а второй --- за 5 ч. Какая часть пути будет отделять велосипедистов друг от друга через 2 ч после выезда?

\setcounter{number}{372} \nast

Сумма трёх чисел равна $64\frac45$; первое число в $2\frac12$ раза, а второе в $3\frac14$ раза больше третьего. Найти эти числа.

\setcounter{number}{388} \n

В трёх мешках всего 460 яблок. Число яблок в первом мешке составляет $\frac34$ числа яблок второго, а в третьем --- в $1\frac12$ раза больше яблок, нежели в первом. Сколько яблок в каждом мешке?

\setcounter{number}{394} \n

Четвероклассник прочитал сначала $\frac{4}{15}$ всей книги, потом $\frac49$ остатка. После этого оказалось, что он прочитал на 25 страниц больше, чем ему осталось читать. Сколько страниц ему осталось прочесть?

\setcounter{number}{416} \n

Расстояние между двумя пристанями по течению катер проходит за 8 часов, а плот~--- за 72 ч. Сколько времени потратит катер на такой же путь по озеру?

\setcounter{number}{712} \nast

Сумма двух чисел равна 640. Если большее число разделить на меньшее, то в частном получится 7 и в остатке 64. Найти эти числа.

\setcounter{number}{236} \n

При каких натуральных значениях $m$ дробь $\,\frac{28}{3m+1}\,$ будет неправильной?

\setcounter{number}{235} \nast

При каких натуральных значениях $k$ дробь $\,\frac{k^2-5}{11}\,$ будет правильной?

\setcounter{number}{334} \nast

Птичка улетает и прилетает в гнездо с кормом каждые 3 минуты. Без корма она летит со скоростью 15 м/с, а с кормом --- 12 м/с. Далеко ли она летает за кормом?

\setcounter{number}{374} \nast

Сумма трёх чисел равна $56\frac38$. Первое число больше второго в $1\frac12$ раза, а второе составляет $\frac12$ третьего числа. Найти эти числа.

\setcounter{number}{267} \nast

Сравни дроби: $\frac{22}{35}$ и $\frac{110}{177}$; $\quad\frac{1998}{1999}$ и $\frac{1999}{2000}$.

\setcounter{number}{456} \nast

Бассейн наполняется двумя трубами за 48 мин, если открыть сразу обе трубы. Через одну трубу бассейн может наполниться за 2 ч. Найти объём бассейна, если известно, что за 1 мин через вторую трубу поступает на 50м$^3$ воды больше, чем через первую.

\setcounter{number}{303} \nasst

Докажи, что значение суммы $\,\frac{1}{1001}+\frac{1}{1002}+\ldots+\frac{1}{2000}\,$ больше, чем $\frac{1}{2}$.

\setcounter{number}{318} \nasst

Найти значение суммы $\frac{4}{5\cdot7}+\frac{4}{7\cdot9}+\frac{4}{9\cdot11}+\ldots+\frac{4}{59\cdot61}$\\
(подсказка: лучше не считать результат ни одного произведения).

\setcounter{number}{304} \nasst

Можно ли число 1 представить в виде $\frac{1}{a}+\frac{1}{b}+\frac{1}{c}+\frac{1}{d}$, \\где $a, b, c, d$ --- нечётные натуральные числа? 
Ответ обоснуй.

\vspace{6mm}
{\centering\subsection*{\textcolor{Mulberry}{Часть III:}}}
{\vspace{-2mm}\centering\subsection*{\textit{Десятичные дроби и уравнения с ними. (Номера 36-45)}}}

\setcounter{number}{511} \n

Найти два числа, разность и частное которых были бы равны 5.

\setcounter{number}{475} \n

В числе $0,\!528047169$ вычеркни 4 знака после запятой так, чтобы получилось:\\
1) наибольшее число $\qquad$ 2) наименьшее число.

\setcounter{number}{476} \n

В числе $1037,\!584629$ вычеркни 3 цифры так, чтобы оставшиеся цифры в том же порядке составили: 1) наибольшее число $\;$ 2) наименьшее число.

\setcounter{number}{540} \n

Задуманное число сначала 10 раз увеличили на $0,5$, а затем 10 раз уменьшили на $0,49$ и получили $12,44$. Какое число было задумано?

\setcounter{number}{536} \n

Если из задуманного числа вычесть $1,\!05$, разность умножить на $0,\!8$, к произведению прибавить $2,\!84$, а затем полученную сумму разделить на $0,\!01$, то получится 700. Какое число было задумано?

\setcounter{number}{544} \n

Решить уравнение: $\,2\cdot\left(0,2-0,02:\left(0,002+0,0002\cdot x\right)\right) = 0,3$.

\setcounter{number}{570} \n

Когда турист прошёл $0,\!35$ всего пути, то до половины ему осталось пройти 6 км.\\ Найти длину всего пути.

\setcounter{number}{548} \n

Сумма двух чисел $100,05$. Одно число на $97,06$ больше другого. Найти эти числа.

\setcounter{number}{551} \n

Найти два числа, зная, что первое число больше второго на 9 единиц и в 9 раз.

\setcounter{number}{479} \nast

Найти сумму $\;0,\!01 + 0,\!02 + 0,\!03 + \ldots + 0,\!98 + 0,\!99$.

\vspace{6mm}
{\centering\subsection*{\textcolor{blue}{Часть IV:}}}
{\vspace{-2mm}\centering\subsection*{\textit{Составление уравнений. (Номера 46-55)}}}

\setcounter{number}{336} \n

Пассажирский поезд проходит расстояние между городами за 15 ч. Поезд ``Стрела'', скорость которого на 35 км/ч больше, проходит это же расстояние за 8 часов. Какое расстояние между городами?

\setcounter{number}{697} \n

Cколько раз надо прибавить одновременно к числу 250 по 7, а к числу 205 по 10, чтобы получить одинаковые числа?

\setcounter{number}{720} \nast

Какое число надо отнять от числителя и знаменателя дроби $\frac{29}{64}$, чтобы получить $\frac{2}{9}$?

\setcounter{number}{745} \n

Трое ребят имели поровну орехов. Когда каждый из них съел по 8 орехов, то у всех вместе осталось столько орехов, сколько вначале было у каждого из них. Сколько у них было орехов?

\setcounter{number}{756} \n

Деду 56 лет, а внуку 14.\\
Когда дедушка будет вдвое старше своего внука?

\setcounter{number}{764} \n

У 35-летнего отца 4 сына. Каждый моложе другого на 2 года, причём старшему 8 лет. Когда всем детям вместе будет столько же лет, сколько и отцу?

\n
%forum

В порту стоят яхты с одной мачтой и шхуны с 2 мачтами. Старенький смотритель порта забыл, сколько шхун и сколько яхт находится в порту. Только помнит, что всего пришло ровно 100 кораблей. Помоги ему восстановить данные, если он насчитал в порту всего 146 мачт.

\setcounter{number}{839} \n

В комнате стоят стулья и табуретки. У каждой табуретки 3 ноги, у каждого стула 4 ноги. Когда на всех табуретках и стульях сидят люди, в комнате всего 39 ног. Сколько стульев и табуреток в комнате?

\setcounter{number}{794} \n

3 утёнка и 4 гусёнка весят 2 кг 500 г, а 4 утёнка и 3 гусёнка весят 2 кг 400 г. Сколько весит 1 гусёенок?

\nast
%forum

У Миши в 7 раз больше конфет, чем у Оли. Если Миша отдаст Оле 21 конфету, то конфет у детей станет поровну. Сколько конфет было у Миши и Оли вместе?

\vspace{6mm}
{\centering\subsection*{\textcolor{Periwinkle}{Часть V:}}}
{\vspace{-2mm}\centering\subsection*{\textit{Нестандартные задачи и задачи с подвохом. (Номера 56-77)}}}

\setcounter{number}{799} \n

Как, имея два сосуда ёмкостью 5 л и 9 л, набрать из водоёма ровно 3 л воды?

\setcounter{number}{804} \n

Произведение двух нечётных однозначных натуральных чисел на 7 больше их суммы. \\
Найти эти числа.

\setcounter{number}{662} \n

Найти сумму: $\displaystyle 1 +2 -3 -4 +5 +6 -7 - 8 +\ldots+301+302$.

\nast 
%forum

Eсть 6 карточек с цифрой 2. Нужно, используя все эти карточки, знаки арифметических действий и, если нужно, скобки, получить число 55: $\qquad2\;\;2\;\;2\;\;2\;\;2\;\;2 = 55$

\setcounter{number}{696} \nast

Какие из чисел $-2;\, -1;\, 0;\, 1;\, 2;\, 3$ являются решениями следующего уравнения:\smallskip\\ \hspace*{3cm}$x\cdot x\cdot x\cdot x\cdot x + 3\cdot x\cdot x\cdot x\cdot x + 2\cdot x\cdot x\cdot x - 3\cdot x -2 = 0$?

\setcounter{number}{784} \nast

Может ли в каком-то месяце быть 5 понедельников и 5 четвергов?

\setcounter{number}{785} \nast

В некотором месяце три пятницы были нечетными числами.\\
Какой день недели был 25-ого числа?

\nast
%forum

Девять столбов соединены между собой проводами так, что от каждого столба отходит ровно 4 провода. Сколько всего проводов протянуто между этими десятью столбами?

\nast
%forum

Каждую секунду бактерия делится на две новые бактерии. Известно, что бактерии заполняют стакан целиком ровно за 1 час. За сколько секунд стакан будет заполнен бактериями наполовину?

\nast
%forum

В корзине лежат фрукты. Все, кроме двух, апельсины. Все, кроме двух, яблоки. Все, кроме двух, бананы. Сколько каких фруктов в корзине?

\setcounter{number}{820} \nast

Если на прямой через равные промежутки поставить 10 точек, то они займут отрезок длины $s$, если же поставить 100 точек, то отрезок длины $\mathds{S}$. Во сколько раз $\mathds{S}$ больше $s$?

\nast
%forum

Деревянный куб покрасили снаружи белой краской, каждое его ребро разделили на 5 равных частей, после чего куб распилили так, что получились маленькие кубики, у которых ребро в 5 раз меньше, чем у исходного куба. Сколько получилось маленьких кубиков, у которых окрашена хотя бы одна грань?

\nast 
%forum

Между домиками Лосяша и Бараша растёт елка. От нее 360 м до домика Лосяша и 440 м до домика Бараша. Друзья одновременно отправились от елки каждый к своему домику. Придя домой, каждый сразу пошел обратно. На каком расстоянии от елки они встретятся, если Лосяш ходит со скоростью 60 м/мин, а Бараш – со скоростью 40 м/мин?

\nast
%forum

Вася посчитал число коробочек, в которых один шарик или больше: их оказалось 8; коробочек, в которых больше одного шарика --- 6; больше двух шариков --- 5; больше трех --- 3; больше четырех --- 2. Коробочек, в которых больше пяти шариков, не было. Найти общее число шариков во всех коробочках.

\nast
%forum

Три поросенка за 3 дня построили 3 домика. За сколько дней 6 таких же поросят построят себе 6 таких же домиков?

\pagebreak

\nast
%forum

Однажды на лестнице нашли странную тетрадь. В ней было записано сто утверждений:\\
\begin{itemize}
    \item [\textbf{1:}] "В этой тетради ровно одно неверное утверждение";
    \item [\textbf{2:}] "В этой тетради ровно два неверных утверждения";\\
    $\vdots$
    \item [\textbf{100:}] "В этой тетради ровно сто неверных утверждений".
\end{itemize}
Есть ли среди этих утверждений верные, и если да, то какие? 

\nast
%forum

Миша, Гриша и Алеша собирали грибы. Миша и Гриша вместе набрали 36 грибов. Гриша и Алеша на двоих собрали 42 гриба. А Миша и Алеша вдвоем набрали ровно 50 грибов. Сколько грибов набрал каждый мальчик по отдельности?

\nast
%forum

В деревне A живет 100 школьников, в деревне B живет 50 школьников. Расстояние между деревнями 3 километра. 
В какой точке дороги из A в B надо построить школу, чтобы суммарное расстояние, проходимое всеми школьниками, было бы как можно меньше?

\nasst
%forum

Сколько чётных чисел можно составить, если использовать только цифры 0, 1, 3, 5, причём каждую не более одного раза?

\nasst
%forum

На лужайке гуляют дети. Босоногих мальчиков среди них столько же, сколько обутых девочек. Кого на лужайке больше, девочек или босоногих детей?

\nasst
%arnold

На книжной полке рядом по порядку стоят два тома: первый и второй. Страницы каждого тома имеют вместе толщину 2 см, а каждая обложка --- толщину 2 мм. Червяк прогрыз себе путь по прямой от первой страницы первого тома до последней страницы второго тома. Какой путь он прогрыз?

\nasst

Даны числительные некоторого языка: 12 --– \textbf{duodas},$\,$ 21 --- \textbf{akobis},$\,$ 54 --- \textbf{corodasodubis},\newline 79 --– \textbf{ak kam corbis}. Нужно записать на этом языке числительные$\,$ 24, 39, 49.

{\flushright{(\textcolor{LimeGreen}{подсказка:} можно посмотреть, как пишется число 80 во французском)\\}}

\vspace{1cm}

\renewcommand{\arraystretch}{2.5}
\renewcommand{\tabcolsep}{0.4cm}

\begin{center}
\begin{tabular}{|c|c|c|c|c|c|c|c|c|}
\hline
    \raisebox{0.5ex}{$\!$\textbf{44}} & \raisebox{0.5ex}{$\!$\textbf{39}} & \raisebox{0.5ex}{$\!$\textbf{42}} & \raisebox{0.5ex}{$\!$\textbf{20}} & \raisebox{0.5ex}{$\!$\textbf{58}} & \raisebox{0.5ex}{$\!$\textbf{47}} & \raisebox{0.5ex}{$\!$\textbf{4}} & \raisebox{0.5ex}{$\!$\textbf{51}} & \raisebox{0.5ex}{$\!$\textbf{36}} \\
\hline
    \raisebox{0.5ex}{$\!$\textbf{27}} & \raisebox{0.5ex}{$\!$\textbf{11}} & \raisebox{0.5ex}{$\!$\textbf{54}} & \raisebox{0.5ex}{$\!$\textbf{7}} & \raisebox{0.5ex}{$\!$\textbf{63}} & \raisebox{0.5ex}{$\!$\textbf{13}} & \raisebox{0.5ex}{$\!$\textbf{29}} & \raisebox{0.5ex}{$\!$\textbf{17}} & \raisebox{0.5ex}{$\!$\textbf{21}} \\
\hline
    \raisebox{0.5ex}{$\!$\textbf{2}} & \raisebox{0.5ex}{$\!$\textbf{9}} & \raisebox{0.5ex}{$\!$\textbf{}} & \raisebox{0.5ex}{$\!$\textbf{69}} & \raisebox{0.5ex}{$\!$\textbf{72}} & \raisebox{0.5ex}{$\!$\textbf{61}} & \raisebox{0.5ex}{$\!$\textbf{}} & \raisebox{0.5ex}{$\!$\textbf{10}} & \raisebox{0.5ex}{$\!$\textbf{43}} \\
\hline
    \raisebox{0.5ex}{$\!$\textbf{40}} & \raisebox{0.5ex}{$\!$\textbf{14}} & \raisebox{0.5ex}{$\!$\textbf{71}} & \raisebox{0.5ex}{$\!$\textbf{33}} & \raisebox{0.5ex}{$\!$\textbf{75}} & \raisebox{0.5ex}{$\!$\textbf{35}} & \raisebox{0.5ex}{$\!$\textbf{68}} & \raisebox{0.5ex}{$\!$\textbf{28}} & \raisebox{0.5ex}{$\!$\textbf{5}} \\
\hline
    \raisebox{0.5ex}{$\!$\textbf{25}} & \raisebox{0.5ex}{$\!$\textbf{60}} & \raisebox{0.5ex}{$\!$\textbf{57}} & \raisebox{0.5ex}{$\!$\textbf{74}} & \raisebox{0.5ex}{$\!$\textbf{77}} & \raisebox{0.5ex}{$\!$\textbf{76}} & \raisebox{0.5ex}{$\!$\textbf{65}} & \raisebox{0.5ex}{$\!$\textbf{45}} & \raisebox{0.5ex}{$\!$\textbf{56}} \\
\hline
    \raisebox{0.5ex}{$\!$\textbf{52}} & \raisebox{0.5ex}{$\!$\textbf{6}} & \raisebox{0.5ex}{$\!$\textbf{64}} & \raisebox{0.5ex}{$\!$\textbf{15}} & \raisebox{0.5ex}{$\!$\textbf{34}} & \raisebox{0.5ex}{$\!$\textbf{55}} & \raisebox{0.5ex}{$\!$\textbf{62}} & \raisebox{0.5ex}{$\!$\textbf{31}} & \raisebox{0.5ex}{$\!$\textbf{18}} \\
\hline
    \raisebox{0.5ex}{$\!$\textbf{23}} & \raisebox{0.5ex}{$\!$\textbf{30}} & \raisebox{0.5ex}{$\!$\textbf{}} & \raisebox{0.5ex}{$\!$\textbf{73}} & \raisebox{0.5ex}{$\!$\textbf{67}} & \raisebox{0.5ex}{$\!$\textbf{70}} & \raisebox{0.5ex}{$\!$\textbf{}} & \raisebox{0.5ex}{$\!$\textbf{26}} & \raisebox{0.5ex}{$\!$\textbf{48}} \\
\hline
    \raisebox{0.5ex}{$\!$\textbf{16}} & \raisebox{0.5ex}{$\!$\textbf{38}} & \raisebox{0.5ex}{$\!$\textbf{41}} & \raisebox{0.5ex}{$\!$\textbf{8}} & \raisebox{0.5ex}{$\!$\textbf{59}} & \raisebox{0.5ex}{$\!$\textbf{32}} & \raisebox{0.5ex}{$\!$\textbf{12}} & \raisebox{0.5ex}{$\!$\textbf{66}} & \raisebox{0.5ex}{$\!$\textbf{1}} \\
\hline
    \raisebox{0.5ex}{$\!$\textbf{46}} & \raisebox{0.5ex}{$\!$\textbf{53}} & \raisebox{0.5ex}{$\!$\textbf{24}} & \raisebox{0.5ex}{$\!$\textbf{49}} & \raisebox{0.5ex}{$\!$\textbf{3}} & \raisebox{0.5ex}{$\!$\textbf{19}} & \raisebox{0.5ex}{$\!$\textbf{50}} & \raisebox{0.5ex}{$\!$\textbf{37}} & \raisebox{0.5ex}{$\!$\textbf{22}} \\
\hline
\end{tabular}
\end{center}

\end{document}
